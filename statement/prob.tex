
\providecommand{\tightlist}{\setlength{\itemsep}{0pt}\setlength{\parskip}{0pt}}
\setcounter{secnumdepth}{0}
\setlist{noitemsep, topsep=0pt}

\subsection{Problem Description}\label{problem-description}

We have learned some string-matching algorithms such as Rabin-Karp and KMP in class, which efficiently find substrings within larger strings. However, these algorithms are not designed for scenarios where we need to efficiently store and retrieve multiple strings based on their shared prefixes. For example, when implementing an autocomplete system, searching for strings with a given prefix or organizing a dictionary for efficient lookups becomes crucial. Such tasks require data structures specifically designed for prefix-based operations.

In this problem, we suggest you learn a new data structure called \textbf{Trie} that can address prefix searching more efficiently. The following description of the \textbf{Trie} might be helpful for you.

\begin{tcolorbox}[colframe = blue, title={Reference Reading}, colback = white, enhanced, breakable, 
    skin first=enhanced,
    skin middle=enhanced,
    skin last=enhanced,
    before upper={\parindent15pt}]

\noindent\textbf{Definitions}

A Trie (pronounced as ``try'') is a specialized tree-based data structure used for storing and retrieving strings, typically in applications where prefix-based operations are crucial. Tries are particularly efficient for tasks such as autocomplete systems, spell checkers, or dictionary management.

Let $T$ be a Trie storing a set of strings $S = \{s_1, s_2, \ldots, s_n\}$, where each string is composed of characters from a fixed alphabet $\Sigma$. 
\begin{itemize}
    \item Each node in the Trie represents a character of the strings in $S$.
    \item Edges between nodes indicate the sequence of characters in the strings.
    \item Strings are stored implicitly through the paths from the root node to the leaf nodes.
\end{itemize}

For example, if $S = \{\text{``cat''}, \text{``car''}, \text{``cart''}, \text{``dog''}\}$, the Trie will branch at shared prefixes like ``ca'' and ``car''.

(figure to be inserted)

\noindent\textbf{Key properties of a Trie}

\begin{itemize}
    \item \textbf{Prefix representation:} Every prefix of a string in $S$ is represented in $T$ by the path from the root to the corresponding node.
    \item \textbf{Search efficiency:} Finding a string or prefix of length $m$ in a Trie takes $O(m)$ time, regardless of the number of strings stored.
    \item \textbf{Space efficiency:} Shared prefixes among strings are stored only once.
\end{itemize}

\noindent\textbf{Implementation}

A Trie is constructed by inserting strings one character at a time. Each node stores a mapping of child nodes for each character in $\Sigma$, and a boolean flag indicating whether the current node marks the end of a valid string. 

To insert a string $s$ into the Trie:
\begin{enumerate}[label=(\arabic*)]
    \item Start from the root node.
    \item For each character $c$ in $s$: If $c$ does not exist as a child of the current node, create a new node for $c$.
    \item Move to the child node corresponding to $c$.
    \item After processing all characters, mark the final node as the end of a string.
\end{enumerate}

To search for a string $s$ in the Trie:
\begin{enumerate}[label=(\arabic*)]
    \item Start from the root node.
    \item Traverse the nodes corresponding to the characters in $s$.
    \item If all characters are found and the final node is marked as the end of a string, $s$ exists in the Trie.
\end{enumerate}

For prefix-based operations, such as finding all strings with a given prefix $p$:
\begin{enumerate}[label=(\arabic*)]
    \item Traverse the Trie to find the node corresponding to the last character of $p$.
    \item Perform a depth-first search (DFS) from this node to collect all strings that share the prefix $p$.
\end{enumerate}

\end{tcolorbox}

In this problem, we ask you to administrate an initially empty dictionary set using Trie to support the following six operations:
\begin{itemize}
    \item \textbf{Insert} operation: Insert the query string into the dictionary set.
    \item \textbf{Check} operation: Check if the query string is in the dictionary set or not.
    \item \textbf{Prefix\_Search} operation: Find the number of strings in the dictionary set that have the query string as a prefix.
    \item \textbf{LCP} operation: Find the string in the dictionary set that has the \textbf{longest common prefix} with the query string. If there are multiple answers, return the string with the smallest lexicographical order.
    \item \textbf{Score} operation: Find the score of the query string, where the score is defined as the total length of the common prefix of all strings in the dictionary set and the query string.
    \item \textbf{Compress} operation: Compress the whole dictionary set by replacing each string in the dictionary set with its non-empty prefix such that all new strings are still unique. Return the smallest possible total length of the new strings. For example, assume that there are four strings ``hsinmu'', ``hsin'', ``hsuantien'', and ``hello'' in the dictionary set. After the Compress operation, ``hsinmu'' can be compressed to ``hsi'', ``hsin'' can be compressed to ``hs'', ``hsuantien'' can be compressed to ``h'', ``hello'' can be compressed to ``he''. The four compressed strings will be ``hsi'', ``hs'', ``h'', and ``he'', which are all unique, and the total length is 8. The Compress operation will only be called once and always be the last one, i.e., only the last operation will possibly be Compress.

\end{itemize}

\subsection{Input}\label{input}

The first line contains one integer $Q$, representing the total number of operations. Each of the next $Q$ lines is given in one of the following formats:
\begin{itemize}
    \item \texttt{1 str}: Indicating an \textbf{Insert} operation with string \texttt{str}.
    \item \texttt{2 str}: Indicating a \textbf{Check} operation with string \texttt{str}.
    \item \texttt{3 prefix}: Indicating a \textbf{Prefix\_Search} operation with string \texttt{prefix}.
    \item \texttt{4 str}: Indicating a \textbf{LCP} operation with string \texttt{str}.
    \item \texttt{5 str}: Indicating a \textbf{Score} operation with string \texttt{str}.
    \item \texttt{6}: Indicating a \textbf{Compress} operation.
\end{itemize}

\subsection{Output}\label{output}

\begin{itemize}
    \item For each \textbf{Check} operation, print \texttt{YES} in uppercase English letters in a single line if \texttt{str} is in the dictionary set. Otherwise, print \texttt{NO}.
    \item For each \textbf{Prefix\_Search} operation, print a non-negative integer in a single line representing the number of strings in the dictionary set that have \texttt{str} as a prefix.
    \item For each \textbf{LCP} operation, print a string in a single line representing the string in the dictionary set that has the \textbf{longest common prefix} with \texttt{str}.
    \item For each \textbf{Score} operation, print a non-negative integer in a single line representing the score of \texttt{str}.
    \item For each \textbf{Compress} operation, print a non-negative integer in a single line representing the smallest possible total length of the new strings.
\end{itemize}

\subsection{Constraints}

\begin{itemize}
    \item $1 \leq Q \leq 10^4$
    \item The length of one single input string is not greater than $10^4$.
    \item The total length of input strings is not greater than $3 \times 10^5$.
    \item The inserted strings are unique.
    \item Each string contains only lowercase English letters.
    \item The \textbf{Compress} operation will only be called once and always be the last one.
\end{itemize}

\subsection{Subtasks}

\subsubsection{Subtask 1 (10 pts)}

\begin{itemize}
    \item Only \textbf{Insert}, \textbf{Check}, and \textbf{Prefix\_Search} operations are implemented.
\end{itemize}

\subsubsection{Subtask 2 (25 pts)}

\begin{itemize}
    \item Only \textbf{Insert}, \textbf{Check}, \textbf{Prefix\_Search}, and \textbf{LCP} operations are implemented.
\end{itemize}

\subsubsection{Subtask 3 (25 pts)}

\begin{itemize}
    \item Only \textbf{Insert}, \textbf{Check}, \textbf{Prefix\_Search}, \textbf{LCP}, and \textbf{Score} operations are implemented.
\end{itemize}

\subsubsection{Subtask 4 (40 pts)}

\begin{itemize}
    \item Includes all operations.
\end{itemize}

\newpage

\subsection{Sample Testcases}

\begin{multicols}{2}
\subsubsection{Sample Input 1}
\begin{verbatim}
7
1 hsinmu
1 hsuantien
1 dsa
1 csie
2 dsa
2 ntu
3 hs
\end{verbatim}
\columnbreak
\subsubsection{Sample Output 1}
\begin{verbatim}
YES
NO
2





\end{verbatim}
\end{multicols}

\begin{multicols}{2}
\subsubsection{Sample Input 2}
\begin{verbatim}
8
1 hsinmu
1 hsin
1 hsuantien
1 hello
4 hsinchiji
4 hsuchihmo
5 hell
6
\end{verbatim}
\columnbreak
\subsubsection{Sample Output 2}
\begin{verbatim}
hsin
hsuantien
7
8





\end{verbatim}
\end{multicols}

\subsubsection{Sample Explanation 2}

\begin{itemize}
    \item \texttt{4 hsinchiji} will find \texttt{hsin} as the longest common prefix with length = 4. Note that even if \texttt{hsinmu} has also the longest common prefix with length = 4, it is lexicographically larger than \texttt{hsin}, so the answer should be \texttt{hsin}.
    \item \texttt{4 hsuchihmo} will find \texttt{hsuantien} as the longest common prefix with length = 3.
    \item \texttt{5 hell} will score 1 on \texttt{hsinmu}, score 1 on \texttt{hsin}, score 1 on \texttt{hsuantien}, and score 4 on \texttt{hello}. Hence, the total score is 7.
    \item After the \textbf{Compress} operation, \texttt{hsinmu} can be compressed to \texttt{hsi}, \texttt{hsin} can be compressed to \texttt{hs}, \texttt{hsuantien} can be compressed to \texttt{h}, \texttt{hello} can be compressed to \texttt{he}. The four compressed strings will be \texttt{hsi}, \texttt{hs}, \texttt{h}, and \texttt{he}, which are all unique, and the total length is 8. It can be proved that there isn't any compression method to have a shorter total length.
\end{itemize}

\newpage

\begin{multicols}{2}
\subsubsection{Sample Input 3}
\begin{verbatim}
10
1 cat
1 car
1 cart
1 dog
2 car
3 car
4 california
4 carbon
5 cartoon
6
\end{verbatim}
\columnbreak
\subsubsection{Sample Output 3}
\begin{verbatim}
YES
2
car
car
8
7





\end{verbatim}
\end{multicols}